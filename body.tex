\section{Introduction\label{sec:intro}}

For a long time, it has been known that there is an ambiguity in
defining the action of fields when the spacetime possesses torsion.
The simplest example of this ambiguity is observed in the theory
of electromagnetism, where the field strength, 
\begin{equation*}
  \free{F}_{\mu\nu} = \pa{\mu} A_\nu - \pa{\nu} A_\mu,
\end{equation*}
is not invariant under the minimal coupling \mbox{($\pa{\mu} \mapsto \nab{\mu}$),} since
\begin{equation}
  \free{F}_{\mu\nu} \mapsto {F}_{\mu\nu} = \free{F}_{\mu\nu} + \tors{\mu}{\lambda}{\nu} A_\lambda.
  \label{fmunu-trans}
\end{equation}
The last term in Eq.~\eqref{fmunu-trans} spoils the $U (1)$ gauge
invariance. This observation has driven to think that
gauge fields do not couple to torsion in order to keep the
$U (1)$ gauge invariance, see for example
Ref.~\cite{Hehl:1976kj}.\footnote{Notice that although this has
  been the main idea behind the ambiguity, one could ask torsion
  to be a conserved quantity, just like in the case of the
  electromagnetic current in the interaction $J^\mu A_\mu$.}

It is intriguing that the gauge fields do possess spin but
neither produce nor interact with the torsion. In order to
address a minimal coupling prescription which maintains the
$U (1)$ gauge invariance in presence of torsion, it was
proposed in Ref.~\cite{PhysRevD.17.3141} a model
where the torsion field is associated to an scalar potential
called \emph{tlaplon}, which transforms in such a way under
the $U (1)$ gauge group, that compensates the ill-defined
transformation in the Eq.~\eqref{fmunu-trans}. Furthermore,
in Ref.~\cite{Mukku:1978mz},
the procedure was generalized to include $SU (N )$ gauge
fields, allowing the minimal coupling of Yang-Mills theory
to torsionful gravity, following the same spirit as in
Ref.~\cite{PhysRevD.17.3141}.

However, it has been argued that within the Palatini
first order formalism written in differential forms, the
introduction of such a \emph{tlaplon} field seems to be
unnecessary, since the gauge and diffeomorphism invariance are
naturally guaranteed without its help, see Ref.~\cite{Benn:1980ea}.
This relies on the fact that gauge connections are indeed
1-forms and singlets under the local Lorentz group.

Both previously mentioned approaches for the inclusion
of gauge fields in torsionful gravity, leads to physically
different theories. The interaction between gauge
fields and torsion is still an open subject of research.

\section{Einstein--Cartan theory\label{sec:EC}}

The Einstein--Cartan (EC) theory of gravitation is
a natural generalization of General Relativity,
where the gravitational sector is described by the
Einstein--Hilbert action, but its built with a
connection which possesses an antisymmetric part,
i.e., it has torsion (which is absent in Genaral
Relativity).

Cartan developed the natural framework to deal with
torsionful gravitational theories, known nowadays as
the first order formalism, where the vierbein (a field
which encodes the information contented in the metric) and
the (Lorentz) connection are considered as independent
fields~\cite{Cartan1922,Cartan1923,Cartan1924,Cartan1925}.

The vierbein field, $\vi{a}{\mu}$, is defined through
its relation with the metric fields, say
\begin{equation}
  g_{\mu\nu} = \eta_{a b} \, \vi{a}{\mu} \vi{b}{\nu},
\end{equation}
where $\eta$ is the Minkowski metric with signature
\mbox{$\eta_{a b} = \diag(-,+,+,+)$,} greek and latin indices
denote spacetime and Lorentz indices respectively.

On the other hand, the Lorentz connection $\spi{\mu}{a}{b}$,
is introduced in order to define the covariant differentiation
under the local Lorentz group.

With these fields one defines the corresponding vielbein
and Lorentz connection one-forms, \mbox{$\vif{a} = \vi{a}{\mu}
  \, \de{x}^\mu$} and \mbox{$\spif{a}{b} = \spi{\mu}{a}{b}
  \, \de{x}^\mu$} respectively. The structure equations
are expressed in terms of these fields, and define the
torsion and curvature two-forms
\begin{align}
  \df \vif{a} + \spif{a}{b} \vif{b} &= \tf{a} = \frac{1}{2} \tors{m}{a}{n} \, \vif{m} \vif{n},
  \label{strI} \\
  \df \spif{ab}{} + \spif{a}{c} \spif{c b}{} &= \rif{a b}{} = \frac{1}{2} \ri{m n}{a b}{} \, \vif{m} \vif{n}.
  \label{strII}
\end{align}


%%%%%%%%% Conclusions %%%%%%%%%
\section{Conclusions\label{conclusions}}

